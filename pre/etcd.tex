\documentclass{beamer}

\usepackage[utf8]{inputenc}
\usepackage{xeCJK}
\usepackage{url}

%Information to be included in the title page:
\title{etcd: 基本介绍}
\author{李嘉睿}
\institute[]{中间件团队}
\date{2020.9}
\usetheme[]{Madrid}

%
% document begin 
%
\begin{document}

\frame{\titlepage}

\begin{frame}
\frametitle{主要内容}
\tableofcontents
\begin{itemize}
    \item 什么是etcd
    \item 为什么使用etcd
    \item etcd一些实现细节
    \item 如何使用etcd
    \item 一些仍需解决的问题
\end{itemize}
\end{frame}

% frame1:什么是etcd
\begin{frame}
\frametitle{什么是etcd}
etcd是一个\alert{强一致}的分布式\alert{键值存储}。

\begin{block}{强一致}
\begin{itemize}
    \item 基于raft共识算法,单一raft备份组,无分片
    \item \alert{线性读写:} leader分配所有备份节点的全局读写顺序,同时保证读总是能读到最新的值
\end{itemize}
\end{block}

\begin{block}{数据模型:键值存储+版本}
扁平的键空间模拟文件目录结构。
\begin{itemize}
    \item \alert{持久化:} 键值对存储在持久化b+树中,支持对键的范围查找
    \item 内存b树缓存指向键值对的指针
    \item \alert{MVCC:} 每次数据版本更新添加数据增量部分
\end{itemize}
\end{block}
\end{frame}

%
% frame2.1: 为什么使用etcd
%
\begin{frame}
\frametitle{为什么使用etcd}

\begin{exampleblock}{元数据存储}
单一raft备份组,无分片。
\begin{itemize}
    \item 无分片,水平扩展能力较差,支持数GB数据(默认2GB,最大8GB),因此海量数据存储需要使用NewSQL数据库。
    \item 无分片,无需分片备份组之间两阶段提交以及分布式锁的开销,性能更好。
    \item etcd的使用举例:k8s利用etcd存储元数据,tikv的PD利用etcd存储region位置等元数据信息。
\end{itemize}
\end{exampleblock}

\begin{exampleblock}{分布式协调}
开箱即用的分布式协调原语。
\begin{itemize}
    \item 提供监视器、租约、leader选举和分布式锁的支持。
    \item 简单易用,支持Restful接口,可以从命令行使用分布式协调服务。
    \item 使用gRPC框架,已经有多种语言的API支持或客户端实现。
\end{itemize}
\end{exampleblock}

\end{frame}

%
% frame2.2: 为什么使用etcd
%
\begin{frame}
\frametitle{为什么使用etcd}

本质上,etcd和zookeeper解决了相同的问题。其比较如下:

\begin{alertblock}{比较}
    \begin{itemize}
        \item 数据模型:etcd使用键值对,支持范围查找,使用role-based访问控制。zk使用树形znode结构,其内包含ACL访问控制列表。
        \item 存储限制:均最多支持几GB数据的存储。
        \item 并发原语:etcd内置并发原语,zk使用外部客户端库curator。
        \item 读操作:zk不支持线性化读,读操作可能读到过时数据。
        \item MVCC:etcd支持MVCC,zk不支持。
        \item 监视器通知:etcd支持范围键值的监视器。
        \item API支持:etcd支持HTTP/JSON API,zk不支持。
        \item RPC框架:etcd使用grpc框架,zk使用自己定制的rpc协议。
    \end{itemize}
\end{alertblock}

\end{frame}

%
% frame2.3: 为什么使用etcd
%
\begin{frame}
\frametitle{为什么使用etcd}

etcd和NewSQL数据库的比较(以TiKV为例)。虽然TiKV最初的Raft实现
是对etcd的Raft实现的Rust版重写,但是这两种kv存储的应用场景
有较大差异。其比较如下:

\begin{alertblock}{比较}
    \begin{itemize}
        \item 实现动机:TiKV为解决传统单机数据库难以(高效)支持分布式(事务)场景的问题,面向分布式数据库存储。etcd专注集群元数据存储以及分布式协调。
        \item 数据模型:相同,均为分布式有序Map。etcd单机存储引擎为boltDB,TiKV单机存储引擎为RocksDB,均和应用程序直接绑定,boltDB使用LSM树作为存储结构而RocksDB使用b+树作为存储结构。
        \item 存储模型:TiKV对数据分片(默认64MB数据为一个Region),一个Region通常有3个或5个备份,存在于一个Raft组中,因此TiKV支持的是multi-raft,需要独立的节点管理组件。
        \item 存储限制:etcd最多8GB,TiKV理论上可以无限水平扩容。
    \end{itemize}
\end{alertblock}

\end{frame}

%
% frame 3.1 etcd的一些实现细节
%
\begin{frame}
    \frametitle{etcd其它细节:raft与etcd的raft实现}
    
    \begin{block}{raft的CP保证}
    \begin{itemize}
        \item Leader选举: 利用quorum解决split-brain问题
        \item 日志备份: 每一个term内,Leader、followers数据的一致性
        \item Leader选举约束: 新旧Leader间数据的一致性
    \end{itemize}
    \end{block}
        
    
    \end{frame}

\begin{frame}
    \frametitle{etcd其它细节:raft与etcd的raft实现}
    \begin{block}{etcd的raft实现:引入learner提升A(vailablity)}
    首先看一下raft的A的问题
    \begin{itemize}
        \item 新成员加入使得leader处于过载
        \item 网络分区,当leader发现失去和quorum的连接,反转回follower
        \item 情形
            \begin{enumerate}
                \item 集群2+2分裂
                \item 2+1分区时加入新节点
            \end{enumerate}
        \item 集群错误配置,加入一个错误的url,并未添加对应的节点,造成节点成员身份改变
    \end{itemize}
    引入learner的好处,所谓learner即不参与投票的集群成员(当learner和leader的日志同步后会转为follower)
    \begin{itemize}
        \item learner不处理读写,即不路由客户端请求到leader
        \item learner日志和leader一致之后,向leader发出member promote请求
    \end{itemize}
    \end{block}
\end{frame}



%
% frame3.2: etcd其它细节
%
\begin{frame}
\frametitle{etcd其它细节:客户端实现}

\begin{block}{grpc1.0}
客户端为每个endpoint维护一个TCP连接,第一个成功建立连接的endpoint作为"pinned address". 
多个TCP连接有利于更快的故障恢复,但是需要更多资源。
\end{block}

\begin{block}{grpc1.7}
首先尝试连接所有endpoints,维护第一个成功连接的TCP连接。
遇到错误时,由客户端的错误处理器(error handler)决定重连或选择新的服务器地址。
需要维护endpoints状态列表,其中不健康状态的判定存在误判的可能,
即被标记为不健康的节点可能在之后恢复健康但客户端无法获取该信息。
\end{block}

\begin{block}{grpc1.23}
客户端为每个endpoint维护一个TCP连接,通过轮转(round robin)负载均衡,
通过gRPC链式拦截器实现重连.
\alert{仍未解决:}网络分区情况下阻塞,缺少集群健康情况查询服务。
\end{block}

\end{frame}



%
% frame
%

% \begin{frame}
% \frametitle{如何使用etcd}
% \end{frame}


%
% frame5.1:一些思考中的问题和之后的规划
%

\begin{frame}
\frametitle{思考中的问题}
    数据迁移:
    \begin{itemize}
        \item 添加新节点,让一个旧的副本节点(可以是leader,raft有直接的LeaderTransfer命令支持)脱离集群。
        \item 利用learner节点,在不影响(参与投票的)总副本数的情况下创建副本。例:TiKV依靠learner节点,在日志完成同步后,调度集群(PD)发出PromoteLearner和RemoveNode命令,让副本节点脱离集群。
        \item mirror maker: etcdctl内置特性,利用watch实现,数据实时同步,低延迟,但因为不是集群的一个节点,因此不能保证数据修改顺序
        \item \alert{tradeoff}:使用mirror低延迟跨数据中心备份数据,使用learner保持所有历史数据以及修改顺序。
    \end{itemize}
\end{frame}

\begin{frame}
\frametitle{思考中的问题}
    集群监控:
    \begin{itemize}
        \item etcd原生导出metrics接口,使用Prometheus进行监控,Grafana进行监控数据可视化。
    \end{itemize}
\end{frame}

\begin{frame}
\frametitle{思考中的问题}
    利用etcd实现故障时主从服务的切换:
    \begin{itemize}
        \item 基于etcd提供的分布式锁和租约实现。只有持有锁的节点能够提供服务,同时设定锁的租约持有时间。正常情况下主服务器持有锁并周期性续约,在出现故障且租约超时时,备服务器获取锁成为服务的提供者。
    \end{itemize}
\end{frame}

%
% frame5.2: 下一步仍需解决的问题
%

\begin{frame}
\frametitle{下一步仍需解决的一些问题}

\begin{itemize}
    \item 数据增量添加的意义和b+树的联系
    \item 软件事务内存
    \item 基于角色的鉴权
    \item API原语的实际使用
    \item MVCC的源码级别实现及应用
    \item 性能指标
    \item 数据迁移
    \item 双活
    \item 注册中心
    \item 配置中心
\end{itemize}
\end{frame}

%
% frame5.3: 欠缺的一些能力
%

\begin{frame}
\frametitle{欠缺的一些能力}

\begin{itemize}
    \item 对Golang的理解和使用,一些常用的go并发模式
    \item 源码阅读能力
    \item 性能调优方法(go pprof,Vtune)
    \item 其它分布式知识(分布式事务,分布式协调,paxos)
    \item 分布式系统理论和实践的结合
    \item 其它分布式系统(ZK,TiKV,k8s、Docker)
\end{itemize}

\end{frame}

%
% frame5.4: 下一步的规划
%

\begin{frame}
\frametitle{下一步的规划}

\begin{itemize}
    \item golang基础
    \item 学习一些测试方法(go-tpc,etcd自带的benchmark工具)
    \item 学习性能调优
    \item 希望尽快具备解决一些问题的能力
\end{itemize}

\end{frame}

%
% frame6.1: 结束
%

\begin{frame}
\frametitle{谢谢}

ppt和做的一些笔记已经上传到github仓库中。

\url{https://github.com/Willendless/internship}

欢迎随时给出修改建议。

\end{frame}

\end{document}
